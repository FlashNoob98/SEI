
\section{Cicli misti combinati}

Si consideri un impianto turbo gas, a questo viene collegato un impianto a
vapore al fine di migliorare i rendimenti.
%LIBRO LOZZA turbine a gas e cicli combinati

\subsection{Ciclo STIG}
Si aggiunge una caldaia in coda alla turbina nella quale circola dell'acqua che
vaporizza e successivamente entra in camera di combustione, riducendo le
temperature di picco riducendo la formazione di ossidi di azoto, immettono
inoltre nell'impianto una portata aggiuntiva, aumentando la potenza.

Al variare dei parametri fluidodinamici all'interno dell'impianto si avranno
delle variazioni dei punti di equilibrio, la prima macchina a notare la
variazione di portata è la turbina, in coda alla camera di combustione.
La pressione in ingresso alla turbina aumenta, aumentando leggermente il
rapporto di compressione $\beta$ si avvicina il compressore alla linea di
stallo fornita dal costruttore, dunque l'intero impianto va verificato.

In assetto cogenerativo si produce acqua calda ``sanitaria'', il vapore può
essere inviato sia in camera di combustione che all'utenza termica. Ad esempio
in inverno il compressore richiede meno potenza ??

Questa modifica viene sempre effettuata su una turbina bi-albero, al fine di
mantenere il numero di giri corretto al generatore, senza alterare le
prestazioni del ciclo termodinamico.

Questi cicli richiedono però un consumo d'acqua pari a 1-2 kg/Kwh, può essere
iniettata in quattro punti differenti, al fine di produrre cogenerazione,
oppure a monte della camera di combustione, all'interno della camera di
combustione oppure direttamente nella turbina di potenza (bassa pressione).

Approach point: differenza tra la temperatura del vapore acqueo in uscita dalla
caldaia e la temperatura del gas in ingresso.
Pinch point: punto di minima distanza tra la temperatura dell'acqua e dei gas
caldi, solitamente al termine del riscaldamento dell'acqua liquida, all'inizio
del cambio di fase che avviene a temperatura costante, sotto la curva a campana
dell'acqua.

Le macchine rotanti possono essere scalate al fine di variare le potenze senza
cambiare il progetto delle macchine, utilizzando la teoria della similitudine;
questa operazione non è possibile per la camera di combustione che invece deve
rispettare la lunghezza caratteristica legata la tempo di residenza della
combustione, funzione soltanto della reazione chimica.


\section{Ciclo ad iniezione d'acqua}
Si considera un impianto di base di tipo ``ICR'' InterCooledRefrigerated, si
rende più efficace o più ``spinta'' sia l'interrefrigerazione che la
rigenerazione, il loro limite era lo scambio termico, iniettando acqua al
circuito di interrefrigerazione e rigenerazione si ottiene un vapore con più
energia da miscelare con l'aria in uscita dal compressore per entrare poi in
camera di combustione.



