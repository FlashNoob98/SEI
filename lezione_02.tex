
\chapter{Introduzione allo studio delle Macchine a Fluido}

Le macchine a fluido si dividono in due grandi famiglie
\begin{itemize}
\item Dinamiche, come compressori e turbine, a flusso continuo

\item Volumetriche, come i motori a combustione interna o pompe e compressori,
il lavoro è svolto mediante una variazione di volume
\end{itemize}

Le macchine si suddividono ulteriormente in:
\begin{itemize}
\item Macchine motrici
\item Macchine operatrici
\end{itemize}
 e ancora in
\begin{itemize}
\item Macchine termiche (IMT)
\item Macchine idrauliche
\end{itemize}

Nelle macchine termiche è sempre presente un ciclo termodinamico, del calore
che riduce sempre i rendimenti, nelle macchine idrauliche invece, lavorando ``a
freddo'' l'energia restituita è quasi interamente pari a quella fornita dal
fluido.

Rispetto alla direzione del moto del fluido si parlerà di macchine
\begin{itemize}
\item Assiali
\item Radiali
\item A flusso misto
\end{itemize}

Mediante l'equazione di Eulero è possibile...

\section{Macchine termiche}
La macchina termica non è l'impianto termico,
Esempio con gruppo turbogas
%Inserisci immagine turbogas

Il generatore è posizionato solitamente vicino al compressore e non alla
turbina per le più basse temperature, a meno che la turbina non sia
multiassiale.

Il moto del fluido può essere stazionario come ad esempio in una turbina, il
flusso è continuo, in realtà è un'approssimazione; oppure in un flusso non
stazionario come in un motore a combustione interna.

Alcuni fluidi sono comprimibili o lo sono talmente poco da essere ritenuti
incomprimibili.

Un fluido può essere viscoso o non viscoso

Nel caso di grossi impianti le turbine possono essere simmetriche, disposte su
due lati come se fossero due turbine collegate.

Gli impianti motore termici possono essere
\begin{itemize}
 \item Impianti a vapore; 2MW - 1000MW
 \item Turbina a gas; 18kW - 500MW
 \item Motori a combustione interna; 0.1kW - 90MW
 \item Impianti combinati (gas-vapore); $>$ 1000MW
\end{itemize}

Le grandezze in ingresso sono calore (kW), in uscita potenza meccanica (kW).

Le turbine idrauliche hanno invece potenze medie superiori a 100MW.


Formula del rendimento di un ciclo combinato, alla potenza della turbina a gas
si aggiunge la potenza della turbina a vapore:
\begin{equation}
\eta_g = \frac{L_u}{m_c\cdot H_i} =\frac{P_u}{\dot{m}_c \cdot H_i} =
\frac{P+P_{TV}}{\dot{m}_f H_i}
\end{equation}

Si ricorda la definizione di portata massica
\begin{equation}
\dot{m}_c = \frac{dm_c}{dt}
\end{equation}

In passato le temperature massime erano limitate dai materiali della turbina,
non c'era abbastanza calore residuo per alimentare la turbina a vapore, per
questo motivo non si realizzavano cicli combinati.

Il rendimento di un impianto è spesso inferiore se l'impianto non lavora a
massima potenza.
Un ulteriore limite della turbina a gas è la temperatura di ingresso dell'aria,
può essere a volte comodo raffreddare inizialmente l'aria d'ingresso al
compressore.

In un gruppo turbogas, solitamente la turbina ha meno stadi, il compressore
deve eseguir un lavoro ``forzato'' contro la tendenza naturale dell'aria ad
espandere, sono necessari più stadi.
Nella turbina il gas espande, avverrebbe anche in maniera naturale, non è
dunque necessario un elevato numero di stadi.
%museo tecnico Spira germania

Rendimento di un impianto idraulico
\begin{equation}
 \eta_G = \frac{P}{\rho\cdot g \cdot h \cdot Q}
\end{equation}
$Q$ è la portata volumetrica, $\rho$ è la densità e $h$ è l'altezza geodetica.

L'energia prodotta in un impianto sarà l'integrale della potenza, oppure il
prodotto tra la potenza e il fattore di utilizzo $f$
\begin{equation}
 E_{kWh} = \int_0^T P_{\text{effettiva}} dt = f\cdot P_n \cdot T
\end{equation}

\section{Cicli termodinamici}
I cicli possono essere suddivisi in reali e ideali
VEDI TABELLA SLIDE

In un ciclo ideale si effettua una trasformazione isoentropica, ad entropia
costante, in caso contrario si sta riducendo energia meccanica, ovvero si
distrugge l'\textit{exergia}, con conseguente produzione di entropia.

Consumo specifico di calore
\begin{equation}
 C_s = \frac{1}{\eta_G} = \frac{m_c\cdot H_i}{L_u}
\end{equation}

VEDI SIDE CON I RENDIMENTI

Aggiungi rendimento di combustione

Per disegnare un ciclo termodinamico si rappresentano prima le due isobare
divergenti, poi si congiungono i punti con le trasformazioni, il calore di
adduzione, caratteristica del ciclo termodinamico, risente di tutto il ciclo e
 ne influisce il rendimento

Potere calorifico inferiore

Non si usa praticamente mai il potere calorifico superiore


Definizioni di DOSATURA
\begin{itemize}
 \item Rapporto aria / combustibile $ \alpha = \frac{m_a}{m_f}$
 \item Rapporto combustibile / aria $ f= \frac{m_f}{m_a} = \frac{1}{\alpha} $
 \item Rapporto di equivalenza $\varphi = \frac{f}{f_{st}} =
\frac{\alpha_{st}}{\alpha} \left\{
\begin{aligned}
>1\ & \text{Eccesso di combustibile}\\
=1\ & \text{Stechiometrico}\\
<1\ & \text{Eccesso di aria}
\end{aligned}\right.$
\end{itemize}

$\alpha_{st}$ è il rapporto aria / combustibile che si ottiene in caso di
reazione stechiometrica.

Nel calcolo delle masse è necessario inserire l'azoto nella massa d'aria che ha
un peso prevalente nella composizione dell'aria.

Quando si considera il metano come combustibile non bisognerebbe assumere il
metano come elemento singolo mentre il gas naturale realmente utilizzato è
solitamente composto al 90\% da metano e la restante parte da altri gas come
butano e propano.

Indice di emissione di CO2 pari al rapporto tra la massa di CO2 e la massa di
combustibile bruciata.

Oppure si può indicare l'indice di emissione moltiplicato per il consumo
specifico ottenendo i grammi di CO2 rispetto all'energia prodotta.
$$
EICO_{2f} = \frac{m_{CO2}}{m_f}1000 \Rightarrow EICO_{2W} = EICO_{2f}c_{sc}
$$

L'\textit{azoto} non partecipa attivamente alla combustione, non si ossida in
una reazione stechiometrica ma in caso di eccesso di ossigeno si ha comunque
una ossidazione parziale dell'azoto con il rilascio dunque di sostanze
inquinanti in atmosfera ($NO_x$).

\section{Equazione di stato dei gas}

$pV = nRT$

Classificazione dei sistemi termodinamici


Vedi definizione flusso stazionario slide, equazione di bilancio della massa
ecc

Nelle macchine alternative solitamente non è vero che il flusso e la portata
siano costanti in ogni istante del ciclo.

Ricorda che calore e lavoro non sono differenziali esatti!



