
Per capire se una macchina è di grossa potenza si capisce dalla camera di
combustione, se fosse una turbina aeronautica dovrebbe essere a sezione
inferiore, il combustore si dice interno, altrimenti con combustore ad anello
la turbina non può essere aeronautica.

\section{Rigenerazione}
La turbina ha un lavoro utile relativamente basso a causa della potenza
necessaria a muovere il compressore.

Si innalza la pressione allo scarico della turbina se si aggiunge il
rigeneratore, si parla di contropressione allo scarico, dunque l'espansione
avviene in un salto entalpico inferiore, dunque si riduce leggermente la
potenza alla turbina. In impianti più complessi gli scambiatori possono essere
molteplici.
La temperatura del flusso in uscita è ancora rilevante, viene sfruttata per
riscaldare l'aria in uscita dal compressore, preriscaldando il flusso prima
della camera di combustione, dunque si risparmia combustibile, ovvero la Q_1 di
adduzione del calore.

Rendimento con rigenerazione, se la temperatura allo scarico della turbina è
molto più alta di quella dello scarico del compressore si può avere
interrefrigerazione.
Il rapporto $\beta$ massimo che massimizza il lavoro è pari a quello che si
otterrebbe con le temperatura $T2=T4$.

L'area sottesa alla curva $4r-r$ rappresenta il calore fornito dai gas caldi al
gas in camera combustione,
$Q_1=C_p(T_3-T_{2R})$
La temperatura di adduzione inoltre è aumentata.



Riducendo Beta (riducendo la taglia della turbina) invece si aumenta la
temperatura,

$$
\eta_d = 1- \frac{1}{\beta^\lambda}
$$


Calore recuperato
$$
|Q_{2r}| = C_p(T_4-T_r)
$$


Rendimento ideale del ciclo con rigenerazione:
$$
\eta_{id} \frac{L}{(Q_1)}
$$

Ha senso fare la rigenerazione con valori di $\beta$ piccoli, quindi macchine
piccole, va calcolato il $\beta$ massimo che massimizza il lavoro, se il
rapporto fosse superiore a questo valore si evita la rigenerazione.



\section{Interrefrigerazione}
Si utilizza questa tecnica per ridurre il lavoro specifico assorbito dal
compressore, spezzando la compressione tra macchine successive.
Viene chiamato anche intercooler lo scambiatore posto tra i due compressori,
viene utilizzato un fluido esterno come l'acqua per raffreddare l'aria un
uscita dal primo compressore.

Con l'interrefrigerazione si è ridotto il lavoro richiesto per la compressione,
il rendimento però è peggiorato, è necessario fornire più calore per riscaldare
il fluido fino al punto 3, per questo motivo può essere utile utilizzare la
rigenerazione e recuperare il calore della turbina per aumentare il rendimento,
il punto 2 infatti si trova ora molto più in basso rispetto al punto 4.

Nel ciclo reale, il ciclo aggiuntivo dovuto all'interrefrigerazione è uno
pseudo ciclo, ha la curva di raffreddamento ad entropia negativa, aumenta
dunque il rendimento, è come se fosse un ciclo ideale deformato.

