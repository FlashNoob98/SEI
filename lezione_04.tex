
\section{I cicli della turbina a gas}
L'impianto turbogas è composto da tre componenti, compressore, camera di
combustione e turbina, può essere presente un circuito di recupero dell'aria.

Tutte le grosse macchine sono di tipo assiale multistadio, è un impianto molto
flessibile, può lavorare con molte tipologie di combustibili diversi.

Le turbine possono essere monoalbero o multialbero, la seconda configurazione
permette al generatore elettrico di mantenere la sua velocità imposta dalla
frequenza di rete senza rischiare di far stallare il compressore.
Nonostante le rappresentazioni grafiche il generatore è in realtà collegato al
compressore e non alla turbina a causa delle più basse temperature.

Il ciclo è il ciclo termodinamico, il circuito rappresenta come è costruito
l'impianto.
Alcuni impianti a circuito chiuso possono utilizzare l'elio anziché l'aria.
Il compressore va regolarmente lavato, sia durante il funzionamento che durante
manutenzioni straordinarie, questo per preservare le prestazioni dell'impianto.

I lavori specifici di un impianto turbogas sono comunque solitamente molto
bassi con l'aria, per questo può essere comodo cambiare il fluido dell'impianto
ed utilizzare ad esempio l'elio. Il compressore ad esempio assorbe gran parte
del lavoro prodotto dalla turbina, rispetto ad una pompa presente in un
impianto a vapore, utilizzando l'interrefrigerazione è possibile ridurre il
lavoro del compressore.


\subsection{Grandezze caratteristiche}
\begin{itemize}
 \item Rapporto di compressione $\beta = \frac{p_2}{p_1} = \frac{p_3}{p_4}$
 \item Rapporti di temperature $\theta = \frac{T_3}{T_1}$ rapporto tra la
massima e minima temperatura
\end{itemize}
Se si considera un ciclo ideale si può utilizzare l'equazione di stato dei gas
$$
\frac{p}{\rho} = RT
$$

Se il gas è ideale i calori specifici sono costanti, dipendono da $R$ e $k$,
$k=\frac{c_p}{c_v}$ è l'esponente dell'adiabatica, dipende dal gas in uso,
mentre $R=c_p-c_v$

Per il calcolo del salto entalpico si usa la seguente relazione
$$
\Delta h = Q - L = cp\Delta T
$$


Il ciclo ideale è quello di riferimento, tutte le trasformazioni adiabatiche
sono legate da
$$
\frac{dp}{dv} = -j \frac{v}{?}
$$

%Inserisci il lavoro di compressione.
$$
|L_C|= d\int_1^2 VDP = \frac{k}{k-1}1V_1(\beta^\lambda
-1)
$$

L'adduzione del calore durante la combustione invece
$$
L_t = \int_3^4 = \frac{k}{k-1z}p_3v
$$

La temperatura media di un impianto turbogas è relativamente alta mentre la
temperatura di sottrazione non è facilmente modificabile, dipende dalla
temperatura ambiente ma dipende dalla temperatura di scarico dei gas combusti,
solitamente anche a $500^\circ$


Il lavoro utile è dato dalla differenza tra il lavoro prodotto dalla turbina e
quello assorbito dal compressore.

Il rendimento è funzione solo di $\beta$ nel ciclo ideale, solitamente però il
rapporto $\beta$ dipende dal tipo di applicazione.

Se si usa un impianto a circuito chiuso con elio si massimizza il lavoro con
una macchina più piccola, con rapporto di compressione inferiore.

Il $C_p$ dell'aria è solitamente circa 1, in realtà per la turbina il $C_p$ dei
gas esausti è solitamente maggiore.

